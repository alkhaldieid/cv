% Created 2023-01-09 Mon 23:52
% Intended LaTeX compiler: pdflatex
\documentclass[11pt]{article}
\usepackage[utf8]{inputenc}
\usepackage[T1]{fontenc}
\usepackage{graphicx}
\usepackage{longtable}
\usepackage{wrapfig}
\usepackage{rotating}
\usepackage[normalem]{ulem}
\usepackage{amsmath}
\usepackage{amssymb}
\usepackage{capt-of}
\usepackage{hyperref}
\author{Eid Alkhaldi}
\date{\today}
\title{Research Statement}
\hypersetup{
 pdfauthor={Eid Alkhaldi},
 pdftitle={Research Statement},
 pdfkeywords={},
 pdfsubject={},
 pdfcreator={Emacs 28.2 (Org mode 9.5.5)}, 
 pdflang={English}}
\begin{document}

\maketitle
As a researcher in the field of artificial intelligence and its applications to science and engineering, I am committed to tackling complex and challenging problems that have the potential to make a significant impact on society. My research aims to advance the state of the art in AI and to develop novel methods and technologies that can be applied to a wide range of scientific and engineering domains.

One of my key areas of research is histological image classification, where I have made important contributions to the development of ensemble optimization. My work in this area has resulted in several high-impact publications, including “Ensemble Optimization for Invasive Ductal Carcinoma (IDC) Classification Using Differential Cartesian Genetic Programming,” in IEEE Access which is a Q1 journal, “Ensemble Optimization Using Clonal Selection Algorithm for Breast Cancer Histology Image Classification”, “Adaptive PSO-Based Ensemble Optimization for Histology Image Classification”, and “Genetically Optimized Heterogeneous Ensemble for Histological Image Classification”. I have also collaborated with researchers from the University of Toledo to advance the understanding of AI-based efficient resources allocation for 5G-Networks and to develop new methods and technologies.

In addition to histological image classification, I am also interested in exploring the intersection of artificial intelligence and various Engineering fields. I believe that this intersection has the potential to unlock new insights and capabilities in both fields, and I am eager to contribute to the development of these emerging research areas. My current work in this area includes the development of Artificial Intelligence models for 5G-networks, which has the potential to improving resources allocation for d2d clients.

At KAUST, I see the potential to further advance my research in artificial intelligence and its applications to science and engineering, and I am excited about the opportunity to join the faculty and contribute to the research community at the university. I believe that the resources and expertise available at KAUST, combined with the diverse and collaborative research environment, would provide an ideal platform for my continued growth as a researcher. I am confident that my research would align well with the research priorities of KAUST and would make a valuable contribution to the university's mission of advancing scientific knowledge and solving global challenges.
\end{document}